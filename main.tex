\documentclass{article}

\usepackage{amsmath} %para el formato de las fórmulas
\usepackage{array} % Para tablas avanzadas
\usepackage{authblk}
\usepackage[spanish]{babel}
\usepackage{biblatex}
\usepackage{booktabs}
\usepackage{csvsimple}
\usepackage{enumitem}
\usepackage{float}
\usepackage[total={18cm,24cm},centering]{geometry}
\usepackage{gensymb}
\usepackage{graphicx}
\usepackage{hyperref}
\usepackage[utf8]{inputenc}
\usepackage{makecell}
\usepackage{multicol}
\usepackage{subcaption}
\usepackage[dvipsnames]{xcolor} %para subrayar de colores
\usepackage{xparse}


% ##############################
% ##-> Settings <- #################
% Figures folder ----------------------------
\graphicspath{{figures/}}
% ---------------------------------------------

% link colors ---------------------------------
\hypersetup{
    colorlinks=true,
    linkcolor=blue,
    filecolor=blue,      
    urlcolor=blue,
    citecolor=blue
    }
% -----------------------------------------------

% Bibliography -------------------------------
\addbibresource{src/referencias.bib}
% -----------------------------------------------

% Figures --------------------------------------
\NewDocumentCommand{\myFigure}
    {O{} m O{} O{0.5}}{
        \begin{figure}[H]
            \centering
            \includegraphics[width=#4\textwidth]{#2}
            \caption{#3}
            \label{#1}
        \end{figure}
}
\numberwithin{figure}{section}
% -----------------------------------------------

% Tables ---------------------------------------
\makeatletter
\csvset{
  autotabularcenter/.style={
    file=#1,
    after head=\csv@pretable\begin{tabular}{*{\csv@columncount}{c}}\csv@tablehead,
    table head=\toprule\csvlinetotablerow\\\midrule,
    late after line=\\,
    table foot=\\\bottomrule,
    late after last line=\csv@tablefoot\end{tabular}\csv@posttable,
    command=\csvlinetotablerow},
}
\makeatother
\NewDocumentCommand{\myTable}{O{} m O{}}{
    \begin{table}[H]
        \centering
        \csvloop{autotabularcenter={#2}}
        \caption{#3}
        \label{#1}
    \end{table}
}
\numberwithin{table}{section}
% -----------------------------------------------

% Equation -------------------------------------
\numberwithin{equation}{section}
% -----------------------------------------------

% ##############################
% #############################

\title{Title}

\author[1]{Luis Daniel Díaz}
\affil[1]{
    Instituto de Física, Facultad de Ciencias Exactas y Naturales, Universidad de Antioquia
    }

\begin{document}

    \maketitle
    \begin{multicols}{2}
        
        \section{Punto 3.}
        \subsection{Deducción de la ecuación diferencial.}
    Cuando el líquido toca un plano infinito, estamos asumiendo que la curvatura del liquido en ese plano es cero. Además, desde esta perspectiva, $\kappa_0 = 0$, de modo que la ecuación que deducimos en clase queda:

    \begin{align*}
        \kappa_1 + \kappa_2 &= 2\kappa_0 + \frac{z}{R_c^2}\\
        \kappa_1  &= \frac{z}{R_c^2} \\
        \kappa_1 &= - \frac{\partial}{\partial z} (1 + z'^2)^{-1/2}
    \end{align*}

    esta última igualad también deducida en clase. Dado que al igualar, la ecuación diferencial sólo depende de $z$, se convierte en una ecuación diferencial ordinaria.

    \begin{align*}
        -\frac{d}{dz}(1 + z'^2)^{-1/2} &= \frac{z}{R_c^2} \\
        d(1 + z'^2)^{-1/2} &= -\frac{z}{R_c^2} dz \\
        \int_{0}^{(1+z'^2)^{-1/2}} d(1 + z'^2)^{-1/2} &= \int_{0}^{-\frac{z}{R_c^2}} dz 
    \end{align*}
    así:
    \begin{align*}
        (1 + z'^2)^{-1/2} - 1 &= -\frac{1}{2} \frac{z^2}{R_c^2} \\
        (1 + z'^2)^{-1/2} &= 1 -\frac{1}{2} \left(\frac{z}{R_c}\right)^2 \\
        1 + z'^2 &= \left[ 1 -\frac{1}{2} \left(\frac{z}{R_c}\right)^2 \right]^{-2} \\
    \end{align*}

    \begin{equation}
        \boxed{z'^2 = \left[ 1 -\frac{1}{2} \left(\frac{z}{R_c}\right)^2 \right]^{-2} - 1 }
    \end{equation}

\subsection{Implementación en python.}

    Se realizo en interactivo adjunto. Puede acceder a él mediante los siguientes links:

    

    \end{multicols}


\end{document}