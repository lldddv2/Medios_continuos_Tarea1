\myFigure[fig:ilustración]
    {figures/punto2.jpg}[Ilustración del problema.][.4  ]

La figura \ref{fig:ilustración} muestra un esquema del problema. 

Sea $K$ el número de partículas en el disco superior, hallemos $K$.

Sabemos que el volumen de la semiesfera será, por definición, $V = V_{mol}N = L_{mol}^3 N$, y genéricamente: 

\begin{align*}
    V &= \frac{1}{2} V_{esf} \\
    &= \frac{1}{2} \frac{4}{3} \pi r^3\\ 
    &= \frac{2}{3} \pi r^3 \\
    &= L_{mol}^3 N 
\end{align*}

Si despejamos el radio de la esfera, obtenemos:

\begin{equation}
    r = (3/2\pi)^{1/3} L_{mol} N^{1/3} \label{eq:r_vol}
\end{equation}

Por otra parte, podemos asumir que $L_{mol}^2 K = A$, donde $A$ es el área del disco superior. Así:

\begin{align*}
    A &= \pi r^2 \\
    L_{mol}^2 K &= \pi r^2  
\end{align*}

y despejando el radio:

\begin{equation}
    r = \sqrt{\frac{1}{\pi}} L_{mol} \sqrt{K} \label{eq:r_area}
\end{equation}

Igualando las dos expresiones para $r$, \ref{eq:r_vol} y \ref{eq:r_area}, obtenemos:

\begin{align*}
    \sqrt{\frac{1}{\pi}} L_{mol} \sqrt{K} &= (3/2\pi)^{1/3} L_{mol} N^{1/3} \\
\end{align*}

\begin{equation}
    \sqrt{K} =  (3/2\pi)^{1/3} \sqrt{\pi} N^{1/3} \label{eq:sqrt_K}
\end{equation}

Ahora, sabemos que $\Delta N = \Delta (pK)$, que por la aproximación de $N$ y $K$ grandes y $p \ll 1$, podemos tratar como una distribución de Poisson. Así:

\begin{align*}
    \Delta N &= \sqrt{pK} \\
    &= \sqrt{p} \sqrt{K} \\
    &= \sqrt{p} (3/2\pi)^{1/3} \sqrt{\pi} N^{1/3} \\
\end{align*}

Por tanto, podemos ver que $\Delta N = C N^{1/3}$, donde $C = \sqrt{p} (3/2\pi)^{1/3} \sqrt{\pi}$.
